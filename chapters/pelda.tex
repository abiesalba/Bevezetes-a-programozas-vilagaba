\chapter{Példa} % Developer guide
\label{ch:impl}

Ez a fejezet az olvasó számára nem releváns. A dokumentum szerkesztés alatt van, ezek csak számomra tartalmaznak LaTex-es példákat pl. táblázatokra vagy prógramkód formátumokra. :)

\begin{lstlisting}
int 5 = 2;
\end{lstlisting}
\begin{center}
	\begin{longtable}{ | m{0.20\textwidth} | 	m{0.30\textwidth} |m{0.40\textwidth}| }
		\hline
		\textbf{Én mint \newline \textit{(As a)}} & \textbf{Szeretnék \newline \textit{(Want)}} & \textbf{Azért hogy \newline \textit{(So that)}} \\
		\hline \hline
		
		Vendég & Játékok listázása. & Azért, hogy láthassam azokat a kalandjátékokat, amelyekhez megtekintési joggal rendelkezem. \\
		\hline
		
		Vendég & Regisztráció a weboldalra. & Azért, hogy igénybe vehessem az alkalmazás azon szolgáltatásait is, amelyekhez autentikáció szükséges. \\
		\hline
		
		Vendég &  Bejelentkezés a weboldalra. & Azért, hogy elérhessem korábban felvett, egyéni adataimat, fiókomat.\\
		\hline
		
		Bejelentkezett felhasználó & Kijelentkezés a weboldalról. & Azért, hogy a böngészőm ne tárolja bejelentkezésem adatait.\\
		\hline
		
		Bejelentkezett felhasználó & Regisztráció törlése.& Azért, hogy korábban felvett egyéni adataim ne legyenek többé elérhetőek, a weboldal adatbázisából törlődjenek.\\
		\hline
		
		Bejelentkezett felhasználó & Fiók adatok módosítása. & Azért, hogy módosíthassam a regisztráció során megadott adataimat.\\
		\hline
		
		Bejelentkezett felhasználó & Új kalandjáték létrehozása. & Azért, hogy új kalandjátékkal bővítsem saját játékaimat.\\
		\hline
		
		Bejelentkezett felhasználó & Már meglévő kalandjáték módosítása. & Azért, hogy módosíthassam a korábban létrehozott kalandjátékom adatait.\\
		\hline
		
		Bejelentkezett felhasználó & Saját kalandjáték törlése. & Azért, hogy törölhessem korábban létrehozott kalandjátékom adatait. \\
		\hline
		
		Bejelentkezett felhasználó & Játékmenet indítása. & Azért, hogy játszhassak a számomra elérhető kalandjátékok valamelyikével.\\
		\hline
		
	\caption{Felhasználói történetek.}
	\label{user_storys}
	\end{longtable}
\end{center}

