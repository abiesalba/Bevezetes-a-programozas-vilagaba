\chapter{Előszó} % Introduction
\label{ch:intro}

Ezen anyagok, leírások célja, hogy betekintést mutasson a programozás világába olyanok számára, akik bár nem rendelkeznek programozási előismeretekkel, de érdekli őket a téma, és nem tudják, hogyan is láthatnának neki önállóan.

Az alábbi ismertető feladatok megoldásához ingyenesen letölthető, különösebb gép igénnyel nem rendelekező eszközökre lesz majd szükséged.

Az alábiakban a c++ programozási nyelvet és a code::blocks fejlesztői környezetet fogjuk használni.

A c++-t azért tartom alkalmasnak erre a célra, mert bár egy roppant bonyolult és sokszor nehezen érthető nyelvről van szó, azonban ez egy ún. multiparadigmás programozási nyelv (ennek jelentését később tárgyaljuk), mely lehetővé teszi, hogy egészen egyszerű kis programokat is írhassunk segítségével, nincs szükség hozzá semmilyen mélyebb tudásra, nincs semmi olyan, ami megbonyolítaná egy teljesen kezdő számára a használatot. Ugyan akkor egy nagyon populáris programozási nyelvről van szó, így akár a későbbiekben is hasznos tudás lehet ennek ismerete. 

A code::blocks fejlesztői környezetet azért tartom alkalmasnak kezdők számára, mivel a telepítését követően viszonylag könnyen használható, segítségével fordíthatjuk és futtathatjuk is a programunkat néhány kattintással, így kezdetben nem szükséges konzolból végezni ezeket, koncentrálhatunk a programozás alapjainak elsajátítására.

Célunk, hogy akár egy emelt szintű informatikai érettségire is fel lehessen készülni ezen anyagok alapján, vagy legalábbis egyfajta ugródeszkaként szolgáljon a programozáshoz, annak elkezdéséhez.